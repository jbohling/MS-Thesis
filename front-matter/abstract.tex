\chapter*{Abstract}\label{ch:abstract}
Steam reforming of hydrocarbons is an important process for the production of hydrogen gas for industrial needs, such as ammonia synthesis and hydrocracking. Due to the high temperatures (\numrange[range-phrase=--]{1300}{1650}\textdegree{}F) and pressures required for reforming, reformer furnace components require materials with excellent creep properties and thus highly alloyed austenitic stainless steels are typically employed. For the outlet manifolds in reformer furnaces, a cast, heat-resistant stainless steel with the composition 20Cr-32Ni-1Nb (ASTM A351 Grade CT15C) is widely used. However, after long-term service exposure this alloy exhibits problems with liquation cracking in the base metal \gls{haz} during repair welding. In the work presented herein, two heats of material from centrifugally-cast manifold components were evaluated to quantify the potential susceptibility to \gls{haz} liquation cracking. The weldability of the 20Cr-32Ni-1Nb materials was evaluated using the Gleeble\texttrademark{} hot ductility test to determine the on-heating and on-cooling ductility (percent reduction in area) at various temperatures after exposure to a simulated welding thermal cycle. The as-received materials and selected hot ductility samples were characterized using \gls{olm} and \gls{sem} with \gls{eds} to correlate the hot ductility behavior with microstructural characteristics.

Both 20Cr-32Ni-1Nb heats showed similar hot ductility behavior when tested (\emph{i}) on-heating and (\emph{ii}) on-cooling from the measured \gls{zdt} of 2375\textdegree{}F. The hot ductility curves revealed that both materials exhibited a poor recovery of on-cooling ductility (Class C3 based on the Nippes criteria) after exposure to the \gls{zdt}, with a noticeable \gls{zdr} and a low \gls{drr} on the order of 20\%. The microstructural evaluation revealed that the loss of on-heating and on-cooling ductility was a result of liquation around niobium carbides along the interdendritic boundaries. \gls{eds} analysis did not reveal the presence of significant amounts of Ni-Nb-Si enriched phases (e.g. G-phase) adjacent to the niobium carbides. The observed liquation along the interdendritic boundaries was attributed to constitutional liquation of niobium carbides which were present in the boundary regions. Based on these findings, the two 20Cr-32Ni-1Nb heats are sensitive to \gls{haz} liquation cracking when exposed to a thermal cycle as would be encountered in repair welding.




%Repair welding of heat-resistant stainless steels used in steam reformer furnace components can be problematic if the material is in the service-exposed condition, due to liquation cracking in the base metal heat-affected zone (HAZ).  Service-exposed, cast 20Cr-32Ni-1Nb heat-resistant steel extracted from a reformer outlet manifold was evaluated using the Gleeble hot ductility test to determine its susceptibility to HAZ liquation cracking.  The hot ductility curves revealed that the service-exposed material exhibited poor ductility recovery when tested upon cooling from the on-heating zero ductility temperature (ZDT) of 2375°F.  On this basis, the service-exposed 20Cr-32Ni-1Nb material would be classified as fairly susceptible to HAZ liquation cracking when exposed to a welding thermal cycle.