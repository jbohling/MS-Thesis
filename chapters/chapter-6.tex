%!TEX root = ..\ms-thesis.tex
\chapter{Conclusions} \label{ch:conclusions}
Two centrifugally-cast, 20Cr-32Ni-1Nb hydrogen reformer outlet manifold components were evaluated using the Gleeble\texttrademark{} hot ductility test to determine the potential susceptibility to \gls{haz} liquation cracking.  The evaluation revealed the following:

\begin{enumerate}
\item Both 20Cr-32Ni-1Nb heats (Cone~1 and Cone~5) exhibited similar as-received microstructures, with numerous NbC carbides along the interdendritic boundaries. A significant extent of fine intradendritic precipitates was also observed in the as-received material. However, evidence of Ni-Nb-Si enriched phases adjoining the interdendritic NbC, commonly reported for service-exposed (“aged”) 20Cr-32Ni-1Nb material, was not observed. Thus, it is likely that both materials in the current study were subjected to a solution-annealing treatment after removal from service, contrary to the information provided with the materials.
\item Both Cone~1 and Cone~5 exhibited Class H1 hot ductility behavior when tested on-heating, with an on-heating \gls{zdt} of \SmartUnit{fahrenheit=2375}. Class H1 on-heating behavior is not, in and of itself, indicative of high sensitivity to liquation cracking.
\item Based on the Nippes criteria, both Cone materials exhibited Class C3 behavior when tested upon cooling from peak temperatures equal to the \gls{zdt}.  Class C3 behavior corresponds to a generally poor recovery of on-cooling ductility after exposure to the \gls{haz} peak temperature.  This classification of the Cone 1 and Cone 5 on-cooling behavior was further corroborated by observed \gls{zdr}s of 24\,C\textdegree{} (75\,F\textdegree{}) and calculated DRR values on the order of 20\% at \SmartUnit{fahrenheit=2225} for both Cone materials.
\item On the basis of the observed Class-H1/Class-C3 behavior together with the determined values of the \gls{zdr} and \gls{drr} criteria, the Cone 1 and Cone 5 materials were shown to be sensitive to liquation cracking in the base metal HAZ. Furthermore, considering the likelihood that both the Cone~1 and Cone~5 materials were in the solution-annealed condition, the liquation cracking susceptibility of service-exposed material is expected to be on a similar, or more likely higher, level. In any case, hot ductility testing during the early stages of development of the 20Cr-32Ni-1Nb alloy would have revealed the susceptibility to liquation cracking and this information would have been valuable for avoiding the repair welding problems experienced by industry.
\item The Cone~1 and Cone~5 hot ductility samples tested on-heating at \SmartUnit{fahrenheit=2375} showed similar microstructures exhibiting cracking along the interdendritic boundaries decorated with NbC and dissolution of fine intradendritic precipitates near the fracture surface. At this test temperature, corresponding to the \gls{zdt}, the beginning of liquation at some interdendritic boundaries was apparent, and this liquation was the cause of the observed nil ductility behavior at this temperature.
\item In the Cone~1 and Cone~5 hot ductility samples tested on-cooling at \SmartUnit{fahrenheit=2300} after exposure to the \gls{zdt}, cracking along the interdendritic boundaries near the fracture surface was again observed accompanied by dissolution of fine intradendritic precipitates. Compared to the on-heating tests, a significantly greater extent of liquation around crack faces and interdendritic phases was apparent, due to longer exposure time near the peak temperature of the simulated thermal cycle (longer time near the \gls{zdt}). The extensive liquation was responsible for the observed zero ductility behavior at the \gls{drt}.
\item In all of the on-heating and on-cooling hot ductility samples, the composition of the prior liquid present at liquated boundaries was similar to the nominal alloy composition and in particular did not show evidence of Ni or Si enrichment, which indicates that incipient melting of Ni-Nb-Si enriched phases (G-phase) was not responsible for the observed liquation.
\item Due to the low extent (nearly zero) of Ni-Nb-Si enriched phases in both the as-received materials and tested hot ductility samples, together with the presence of Nb in the prior liquid, the observed liquation of interdendritic boundaries at temperatures below the bulk solidus temperature was attributed to constitutional liquation of NbC located on the boundaries.

\end{enumerate}

%{\item The fracture paths in a Cone 5 On-Heating 2375\textdegree{}F (ZDT) sample were observed to proceed along the interdendritic boundaries, and the crack faces were closely associated with the interdendritic secondary phases. }

%and would thus warrant appropriate precautions during welding to prevent the occurrence of cracking
