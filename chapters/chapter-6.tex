%!TEX root = ..\ms-thesis.tex
\chapter{Conclusions} \label{ch:conclusions}
Materials from two centrifugally-cast 20Cr-32Ni-1Nb cones were evaluated using the Gleeble hot ductility test to determine the potential susceptibility to hot cracking.  The evaluation revealed the following:

\begin{enumerate}
\item The Cone 1 and Cone 5 materials exhibited similar as-received microstructures, with numerous intradendritic precipitates as a result of high temperature service exposure.
\item Both Cones exhibited Class H1 hot ductility behavior when tested on-heating, which in and of itself is not indicative of high sensitivity to hot cracking.
\item Both Cones were determined to have an on-heating ZDT of 2375\textdegree{}F.
\item Based on the Nippes criteria \cite{nippes_further_1957}, both Cones exhibited Class C3 behavior when tested upon cooling from peak temperatures equal to the ZDT.  Class C3 behavior corresponds to a generally poor recovery of on-cooling ductility after exposure to the HAZ peak temperature.  This classification of the Cone 1 and Cone 5 on-cooling behavior was further corroborated by observed \gls{zdr}s of 75F\textdegree{} and calculated DRR values on the order of 20\% at 2225\textdegree{}F for both Cones.
\item On the basis of the observed Class-H1/Class-C3 behavior together with the determined values of the \gls{zdr} and DRR criteria, the CT15C Cone 1 and Cone 5 materials would be classified as fairly sensitive to hot cracking in the base metal HAZ and would thus warrant appropriate precautions during welding to prevent the occurrence of cracking.
\item The fracture paths in a Cone 5 On-Heating 2375\textdegree{}F (ZDT) sample were observed to proceed along the interdendritic boundaries, and the crack faces were closely associated with the interdendritic secondary phases.  The identity of these phases remains to be determined.
\item Based on the overall similarity of the as-received (pre-test) microstructures of Cone 1 and Cone 5 and the resulting similarity of the hot ductility behavior (as evaluated by the Nippes \cite{nippes_further_1957}, \gls{zdr}, and \gls{drr} criteria), it appears that the Cone 5 material used in this work was not solution-annealed as had been supposedly indicated.
\end{enumerate}



