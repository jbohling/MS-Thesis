%!TEX root = ..\ms-thesis.tex
\chapter{Future Work} \label{ch:future-work}
\begin{enumerate}
\item Having characterized the hot ductility behavior of 20Cr-32Ni-Nb material which was known to be crack-sensitive (the current work) and thereby established a basis for comparison, the hot ductility behavior of modified chemistry heats of 20Cr-32Ni-Nb should be investigated, for example the modified chemistry suggested by (Hoffman chemistry modifications) which [changes to chemistry]. These modified chemistry materials showed improved mechanical properties after service exposure compared to the typical chemistry and thus the modified chemistry appears promising in terms of reducing the liquation cracking propensity of the alloy.
\item Because it has been observed that cast tee and cast cone materials exhibit different reductions in mechanical properties after service exposure, hot ductility evaluation of cast tee materials should be performed to quantify the cracking propensity of statically cast materials (with potentially larger grain size and greater formation of secondary phases) using criteria that are directly related to welding (rather than more general methods such as tensile testing).
\item Dissolution of bulk samples from as-service exposed and simulated HAZ samples to extract precipitates for XRD analysis to quantitatively evaluate the phase fractions and changes thereof during welding
\item Evaluation of carbon extraction replicas in TEM for quantitative composition of secondary phases using EDXS or electron diffraction
\item TTP data for standard and modified CT15C chemistries to determine over what temperatures the silicide phase forms, is stable, etc... and to see if the modified chemistry shows improved behavior in the long term (\emph{did Hoffman show that modified chemistry resulted in lesser extent of G-phase??})
\end{enumerate}
