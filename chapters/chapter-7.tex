%!TEX root = ..\ms-thesis.tex
\chapter{Future Work} \label{ch:future-work}
The hot ductility behavior of modified chemistry heats of 20Cr-32Ni-Nb, which aim to reduce the formation of Ni-Nb silicide phases upon service exposure, should be investigated to determine if the modified chemistry alloys are effective in improving the repair weldability of service-exposed material. For example, the modified chemistry 20Cr-32Ni-Nb alloy explored by \citet{hoffman_cast_2003}, in which the Si content was reduced and the Nb content was also reduced to a level below the stoichiometric Nb/C ratio of 7.7, was shown to reduce the formation of Ni-Nb silicides after long-term aging with concomitant improvements in tensile properties. \citet{dewar_correlation_2013} used a thermodynamic approach to investigate the effects of nitrogen additions to 20Cr-32Ni-Nb and found that nitrogen decreased the fraction of G-phase and lowered the G-phase stability temperature. While these changes appear promising, weldability testing together with microstructural evaluation must be conducted to verify whether the chemistry modifications are actually effective in reducing the propensity for HAZ liquation cracking during repair welding.

Weldability evaluation of cast tee materials should also be performed to quantify the cracking propensity of statically cast materials (with potentially larger grain size and greater formation of secondary phases). \citet{hoffman_high_2000-1} observed that statically cast tee materials exhibited a greater reduction in tensile properties after service exposure than centrifugally cast cones. This behavior suggests that statically cast materials may show a greater susceptibility to liquation cracking upon repair welding, which should be investigated through hot ductility testing.


% \item Dissolution of bulk samples from as-service exposed and simulated HAZ samples to extract precipitates for XRD analysis to quantitatively evaluate the phase fractions and changes thereof during welding
% \item Evaluation of carbon extraction replicas in TEM for quantitative composition of secondary phases using EDXS or electron diffraction
% \item TTP data for standard and modified CT15C chemistries to determine over what temperatures the silicide phase forms, is stable, etc... and to see if the modified chemistry shows improved behavior in the long term (\emph{did Hoffman show that modified chemistry resulted in lesser extent of G-phase??})
