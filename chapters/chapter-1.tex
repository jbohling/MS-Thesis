%!TEX root = ..\ms-thesis.tex
\chapter{Introduction} \label{ch:introduction}

Steam reforming of hydrocarbons is an important industrial process for the production of hydrogen gas either as an end product or as an input for other processes such as ammonia synthesis and petroleum hydrocracking. In the steam reforming process, a hydrocarbon feedstock is reacted with steam, usually in the presence of a catalyst, to yield hydrogen gas and byproducts according to the following reactions (Nielsen 1984):
\begin{align}
%\ce{C_{n}H_{m}}+\ce{n H2O} -> \ce{n CO}+\bigl(n+\frac{m}{2}\bigr)\ce {H2} \quad (-\Delta{}H^{0}_{298} < 0) \\
\ce{C_{n}H_{m} + n H2O -> n CO + } \bigl(n + \frac{m}{2}\bigr) \ce{H2} \quad (-\Delta{}H^{0}_{298} < 0) \label{eq:1} \\
\ce{CO + H2O <=> CO2 + H2} \quad (-\Delta{}H^{0}_{298} = 41.2 \, \text{kJ mol}^{-1}) \label{eq:2} \\
\ce{CO + 3H2 <=> CH4 + H2O} \quad (-\Delta{}H^{0}_{298} = 206.2 \, \text{kJ mol}^{-1}) \label{eq:3}
\end{align}
Most commonly, methane (CH4) in the form of natural gas is the preferred feedstock, although heavier hydrocarbons such as propane, naphtha, and heptane can be used depending upon availability and cost (Nielsen 1984, Ullmann). In the above ``gasification'' reactions, steam functions as an oxiziding agent to break apart the hydrocarbon; in some variants of the process, air is also added as a further oxidizer. The catalyst is typically nickel-based on an oxide substrate. Steam reforming 

By adjusting the process conditions (exit temperature and amount of steam), the equilibria of the reforming reactions \ref{eq:1}--\ref{eq:3} can be modified to favor certain products and thereby adjust the composition of the product gas. For example, for ammonia synthesis which is one of the largest consumers of industrial hydrogen, the typical desired process conditions are 3.3 MPa exit pressure, 1073 K exit temperature, and a steam to carbon ratio of 3.7 (Nielsen 1984) to minimize the methane content in the product. Considering that reaction \ref{eq:1} is endothermic while reactions \ref{eq:2} and \ref{eq:3} are exothermic, under the process conditions just listed (for minimizing methane) the overall reaction is endothermic and thus external heating is required. The necessary heat is supplied by enclosing the reaction apparatus in a \emph{reformer furnace}.

A schematic of a typical reformer furnace is shown in Figure X. The furnace is a tubular reformer design, in which the feedstock stream (e.g. containing natural gas and steam) is fed simultaneously through a series of identical tubes, containing the catalyst material, which are externally heated. These tubes are visible in Figure X in a vertical arrangement, with the burners installed on the sides of the furnace walls. The inlet side where the preheated feedstock enters the tubes is at the top of the furnace, with the hot product gas mixture collected at an outlet manifold at the bottom of the furnace, consisting of ``hot side'' header, tee, and reducer ("cone") which connects to a refractory-walled transfer line. Through the transfer line, the product gas flow is directed to a waste heat recovery boiler which generates high-pressure steam to be used in the reformer (Nielsen 1984). A photograph of another furnace is shown in Figure Y in which the outlet pigtails (from the reformer tubes), the outlet header and tee, and cone are clearly visible.

Because high exit temperatures maximize the yield of hydrogen from the reformer (Ullmann), typical temperatures are in the range of 700--900C on the outlet side of the reformer furnace. Thus, the reformer tubes and outlet manifold require materials possessing excellent creep performance in addition to resistance to carburization and oxidation. The traditional choice for reformer tubes was centrifugally cast HK-40 (20Cr-25Ni) alloy  (Nielsen 1984), although this has been supplanted by newer alloys such as the HP-Modified alloys (25Cr-35Ni) due to higher creep strength (NiDi reformer tube report). For the outlet manifold components, Alloy 800, HU-40 (19Cr-39Ni), and HK-40 have been used previously (Shibasaki 1993). The HU-40 and HK-40 alloys were implemented because they offered higher creep strength than Alloy 800, but they were discovered to have problems either with excessive thermal gradients (for HU-40), which was the same problem as originally experienced with Alloy 800, or with unacceptable loss of ductility after long-term service exposure (HK-40) (Shibasaki 1993, Collins 1980). To increase the ductilty after service exposure while maintaining high creep strength, a cast steel with the nominal composition of 20Cr-32Ni-Nb was developed (Collins 1980). This alloy, which is specified as Grade CT15C in ASTM A351 \cite{astm_a351_2010}, has similar creep strength to HK-40 (Shibasaki 1993) with improved ductility after service-exposure. 

Despite these improvements, the 20Cr-32Ni-Nb alloy has experienced significant issues with 



% Due to high exit temperatures in the range of 700--900C on the outlet side of the reformer furnace, the reformer tubes and outlet manifold require materials possessing excellent creep performance. 


% Because the efficiency of the reforming reactions are linked to the exit temperature, materials with excellent creep performance are necessary for both the reformer tubes and outlet manifold to enable high operating temperatures to ensure high yield.


% Because the efficiency of the reforming reactions are dependent upon maintaining a high exit temperature, materials with excellent creep performance are necessary for both the reformer tubes and outlet manifold to enable high operating temperatures to ensure high yield.


% how does tube creep properties influence tube design?


% Because the reaction in 1 is endothermic while 2 and 3 are exothermic, the overall heat balance is negative under most process conditions and external heat must be supplied. Thus, steam reforming occurs in a reformer furnace having numerous reformer tubes which contain the catalyst material and through which the reactants are passed. 


