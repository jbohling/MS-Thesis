%!TEX root = ..\ms-thesis.tex
\chapter{Experimental Methods} \label{ch:experimental-methods}

\section{Hot Ductility Tests}
Standard hot ductility samples with dimensions of 114\,mm (4.5\,in) [length] by 6.35\,mm (0.25\,in) [diameter] were extracted from the Cone~1 and Cone~5 base metals adjacent to the respective cone-to-tee welds where cracking during repair welding had been observed.  A thermal cycle characteristic of \gls{smaw} with \SI[round-mode=places,round-precision=2]{2.75}{\kilo\joule\per\milli\meter} (\US{70}{\kilo\joule\per\inch}) energy input in 38\,mm (1.5\,in) stainless steel plate~\cite{nippes_heat-affected_1955}, with an initial plate temperature of \SmartUnit{fahrenheit=80}, was utilized for the on-heating hot ductility tests.  This thermal cycle was determined using the F(s,d) method~\cite{nippes_cooling_1949} based on data provided by Dynamic Systems, Inc.~\cite{duffers_haz_1989}. The peak temperature of the chosen thermal cycle was \SmartUnit{fahrenheit=2450}, equal to the estimated solidus temperature for the CT15C alloy based on data for Incoloy 800H (CT15C is considered a ``cast'' version of 800H) and which would roughly correspond to the maximum temperature experienced at the weld fusion line.

Once the on-heating hot ductility tests had been performed using the aforementioned thermal cycle and the \gls{zdt} had been determined, the thermal cycle was re-scaled so that the peak temperature was equal to the \gls{zdt}. The modified thermal cycle was subsequently used to perform the on-cooling hot ductility tests.  Both the on-heating and on-cooling tests were conducted according to the parameter guidelines recommended by a detailed study~\cite{lundin_standardization_1990_experiment} undertaken to establish a standardized procedure for hot ductility testing.  The specific test parameters utilized in the current work are given in Table~\ref{tab:hot-ductility-parameters}.  For each test, the output of the control thermocouple was recorded, and the post-test cross-sectional area was determined and used to calculate the ductility in terms of percent reduction in area for each specimen.  The collected data were used to create plots of on-heating and on-cooling ductility (\% RA) as a function of test temperature.

\begin{table}[h]
\caption{Parameters and Conditions Used for Hot Ductility Testing, Based on Recommendations in~\citet{lundin_standardization_1990_experiment}}
\begin{tabular}{ lp{4in} }
\toprule
\textbf{Parameter} & \textbf{Condition} \\
\midrule
Thermal Cycle & 38\,mm (1.5\,in) stainless steel plate, \gls{smaw} process, \SI[round-mode=places,round-precision=2]{2.75}{\kilo\joule\per\milli\meter} (\US{70}{\kilo\joule\per\inch}) energy input, \SmartUnit{fahrenheit=80} initial plate temperature \\
\addlinespace
Sample & 114\,mm (4.5\,in) length, 6.35\,mm (0.25\,in) diameter, 1/4-20 thread \\
\addlinespace
On-Cooling Peak Temp. & Zero Ductility Temperature determined from On-Heating Curve \\
\addlinespace
Crosshead Speed & \SI{50}{\milli\meter\per\second} (\US{2}{\inch\per\second}) \\
\addlinespace
Jaw Separation & 16\,mm (0.625 in) \\
\addlinespace
Control Thermocouple & 0.25\,mm (0.01\,in) diameter Chromel-Alumel; Percussion Welding / Separate attachment technique, \SI{1}{\milli\metre} wire spacing \\
\addlinespace
%Preload & \SIrange{3}{4}{\kilo\newton} \\
Test Atmosphere & Air \\
\bottomrule
\end{tabular}
\label{tab:hot-ductility-parameters}
\end{table}

\section{Microstructure Characterization}
Metallographic evaluations were conducted for the as-received Cone~1 and Cone~5 materials, utilizing remnants of the slices removed from the cones for machining of the hot ductility samples.  Metallography was also performed on selected hot ductility samples.  The hot ductility samples were sectioned longitudinally and mounted in castable epoxy and were subsequently ground and polished to a \SI[round-mode=places,round-precision=2]{0.05}{\micro\meter} finish.  The polished samples were electrolytically etched with an aqueous 10\% oxalic acid solution using a stainless steel cathode (\SI{50}{\milli\meter} (\US{2}{\inch}) spacing between the sample and cathode). For the as-received base metal samples, a potential of \SI{6}{\volt} for \SIrange[range-phrase=--]{3}{5}{\second} was used; for the hot ductility samples, it was found that a lower potential, \SI[round-mode=places,round-precision=1]{1.5}{\volt} for \SI{3}{\second}, was necessary to avoid excessive attack of certain phases near the fracture surfaces. \Gls{olm} examination of the etched samples was performed with a Nikon MA-200 inverted metallograph equipped with a 12 megapixel digital camera for obtaining micrographs. \Gls{sem} examination was conducted on selected samples using a LEO Gemini 1525 field emission \gls{sem} equipped with an Oxford Instruments PentaFet \gls{eds} system. Typically, a beam voltage of \SI{20}{\kilo\volt} was used for both \gls{se} and \gls{bse} imaging and for \gls{eds} analysis.