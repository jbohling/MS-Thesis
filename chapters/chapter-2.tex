%!TEX root = ..\ms-thesis.tex
\chapter{Literature Review} \label{ch:literature-review}

\section{Cast Heat-Resistant Austenitic Stainless Steels}


\subsection{Effects of Alloying Elements}
\subsubsection{Chromium}
\subsubsection{Nickel}
\subsubsection{Niobium}
The primary function of niobium (Nb) additions in heat-resistant alloys is to increase the high-temperature creep strength (cite). Niobium combines preferentially with carbon to form hard, highly stable niobium carbides (NbC)
\subsubsection{Carbon}
\subsubsection{Silicon}



\subsection{Metallurgy}

\section{Liquation Cracking} \label{sec:liquation-cracking}
Liquation cracking is a type of weld-related “hot” cracking\footnote{“Hot” cracking occurs at high temperatures during welding, as opposed to “cold” cracking which occurs at or near ambient temperature (e.g. hydrogen related cracking).} which is associated with the formation of liquid films\footnote{In contrast to solidification cracking (another form of hot cracking) in the weld deposit, wherein cracking is associated with liquid films \emph{remaining} along the boundaries at the terminal stages of solidification.} along grain boundaries in the \gls{haz} immediately adjacent to the fusion line (Lippold weldability). Liquation cracking occurs at this location because it is the region of the \gls{haz} immediately adjacent to the fusion line which is subjected to the highest peak temperatures during welding. Although liquation cracking can occur both in the base metal \gls{haz} and in weld metal \gls{haz}s (in the case of reheated weld metal in multi-pass weldments), only the former will be discussed here. In common with other forms of hot cracking, \gls{haz} liquation cracking requires the simulataneous occurrence of two factors: a critical level of applied strain and a susceptible microstructure exhibiting limited ductility over a critical temperature range. The imposition of strain on the \gls{haz} is inherent to welding, either from mechanical restraint (e.g. arising from the weld geometry or from external fixturing) or from thermal contraction during the on-cooling portion of a welding thermal cycle. \citet{yeniscavich_correlation_1970} has estimated the level of thermal strain in the \gls{haz} to be on the order of < 1\%, indicating that an \gls{haz} region must exhibit essentially zero ductility for cracking to occur. 

%Figure showing distribution of strain in the HAZ???

A condition of zero ductility in the \gls{haz} originates from the formation of liquid films along the grain boundaries in the \gls{haz}, since liquid films have very limited capability to support stress (cite). Liquid along the grain boundaries can be formed from several sources. Segregation of impurity elements and/or alloying elements (in solid solution) will occur at grain boundaries, since these regions were the last to solidify and thus will have an inherently lower melting temperature than the adjacent matrix. This is especially true of cast materials where the original dendritic structure has not been disrupted by subsequent working/forming operations. Another potential source of liquation is low-melting point phases which preferentially form at grain boundaries (again, due to segregation). Finally, liquid can also be formed by the mechanism of constitutional liquation as developed by \citet{pepe_effects_1967}. Constitutional liquation is a non-equilibrium phenomenon arising from the rapid heating rates typical of welding and can be explained in a simplified manner in relation to the schematic binary phase diagram shown in Figure X. For a nominal composition C0 which exhibits a two-phase microstructure at low temperatures, during the heating portion of a welding thermal cycle the temperature will increase to the point where the equilibrium phase diagram predicts a single phase region and thus the secondary phase will begin to dissolve. However, due to the rapid heating rate associated with welding, insufficient time is available for complete dissolution of the secondary phase (which must proceed by diffusion), thereby creating a concentration gradient at the interface between the secondary phase and the matrix. Due to the concentration gradient, a region of the matrix enriched in solute will exist at the particle-matrix interface. As the temperature continues to increase rapidly during the heating portion of the thermal cycle, the temperature will eventually exceed the local melting temperature of the enriched region and liquid will form. In the case where the secondary phase exists primarly along grain boundaries (or interdendritic boundaries in a cast material), constitutional liquation of the secondary phase will produce liquid along the boundaries and thereby establish the conditions for liquation cracking. Constitutional liquation of constituent particles in the \gls{haz} has been observed in a number of alloy systems, including liquation of niobium carbide (NbC) in Inconel 718 \cite{radhakrishnan_phase_1991} and AISI 347 (cite WRC 509???).

% In general, explanations of liquation cracking phenomena in the \gls{haz} originate from weld metal hot cracking (i.e. solidification cracking) theories. 

% Liquation cracks vs hot cracks in PMZ of weld HAZ (show diagram)

%Figure for constitutional liquation

\section{Weldability Evaluation} \label{sec:weldability-evaluation}

The term \emph{weldability} as defined by the American Welding Society corresponds to “the capacity of [a] material to be welded under the imposed fabrication conditions into a specific, suitably designed structure performing satisfactorily in the intended service“ \cite{aws_terms_2010}. Thus, the weldability of a material encompasses both the fabrication and in-service performance characteristics of a weldment. In order to evaluate these characteristics, \emph{weldability testing} is performed to ensure that a given combination of material, weld configuration/design, and welding parameters will result in a satisfactory weldment. A number of weldability tests to evaluate susceptibility to hot cracking and liquation cracking (described in Section~\ref{sec:liquation-cracking}) have been developed and are divided into two categories: self-restraint tests and externally-loaded tests \cite{farrar_hot_2005}. As described in Section~\ref{sec:liquation-cracking}, liquation cracking (and hot cracking in general) essentially arises from the inability of a susceptible microstructure to accommodate strain over a particular critical temperature range. In self-restraint tests, the inherent restraint of the chosen weld configuration is used as the source of strain, while in externally-loaded tests, the strain is applied by an external device or instrument. Self-restraint tests are generally considered to be a qualitative, “go/no-go” type of test in contrast to externally-loaded tests which are designed to impose strain under controlled conditions and are thus a more quantitative evaluation method that is more sensitive to variables in materials or welding conditions \cite{farrar_hot_2005}. Among externally-loaded tests for evaluating liquation cracking, two that are widely used are the Varestraint test \cite{lundin_varestraint_1965} and the Gleeble\texttrademark{} hot ductility test \cite{nippes_investigation_1955}. The latter test method was utilized in the current study and a summary of the test method and associated evaluation criteria are described in the following section.

% The requirements of an “ideal” weldability test include the following (DMIC report 165):

% \begin{compactenum}
% \item Direct correlation with actual fabrication or service
% \item High sensitivity to the effects of welding variables
% \item Good reproducibility of results
% \item Simplicity
% \item Economy in the use of materials
% \item Low cost
% \item Applicability to all welding processes
% \end{compactenum}


\subsection{The Hot Ductility Test}
The origin of the hot ductility test can be traced back to work performed by \citet{nippes_cooling_1949} on the measurement of the actual time-temperature history (“thermal cycle”) experienced in the \gls{haz} of various alloys during welding.  Using this data as a foundation, \citeauthor{nippes_development_1949} developed a device, the “Gleeble,” capable of reproducing a given thermal cycle by resistively heating a sample and controlling the temperature as a function of time using an attached fine-wire thermocouple \cite{nippes_development_1949}. This technique enabled the duplication of a specific region of the \gls{haz} in a macroscopically-sized sample suitable for mechanical testing.  The capabilities of the machine were expanded by adding a loading system which permitted the sample to be deformed or fractured at any point in the thermal cycle.  This capability of the Gleeble was put to use in the development of the hot ductility test \cite{nippes_investigation_1955}, in which the ductility of a particular alloy (in terms of percent reduction in cross-sectional area) is determined at various time-temperature points in the thermal cycle.  The hot ductility is normally determined in two distinct modes: “on-heating” and “on-cooling.”  Tests to determine “on-heating” hot ductility are performed during the initial portion of the thermal cycle where the sample is heated rapidly toward the peak temperature (see Figure~\ref{subfig:on-heating-schematic}).  On-heating tests are typically performed at sequentially higher temperatures approaching the calculated \gls{haz} peak temperature (usually in the vicinity of 2400\textdegree{}F) until a test temperature is reached where the measured on-heating ductility drops to zero (0\%~RA).  This temperature is designated the \gls{zdt}.  Once the \gls{zdt} has been determined, “on-cooling” hot ductility tests are performed in the portion of the thermal cycle where the sample temperature is decreasing after exposure to a peak temperature corresponding to the \gls{zdt} (see Figure~\ref{subfig:on-cooling-schematic}).  As will be discussed later, the measured on-cooling ductility behavior is considered the most indicative criteria of a material’s susceptibility to hot cracking.  Figure~\ref{fig:schematic-hot-ductility-curves} shows schematic examples of the types of curves which can be constructed using the data obtained (\%RA vs. test temperature) from the on-heating and on-cooling hot ductility tests just described; important features of these curves are labeled in the plot and will be discussed in more detail in a following section. Further details regarding the history and development of the Gleeble, including references to other development papers not cited here and to research papers utilizing the Gleeble, can be found in the review articles by \citet{savage_apparatus_1962}, \citet{lundin_historical_1997}, and \citet{lundin_standardization_1990_history}. 

\begin{figure}
    \centering
    \subfloat[On-heating hot ductility testing]{\label{subfig:on-heating-schematic}\includegraphics[width=4.5in]{figures/hot-ductility/on-heating-schematic.png}} \\
    \subfloat[On-cooling hot ductility testing]{\label{subfig:on-cooling-schematic}\includegraphics[width=4.5in]{figures/hot-ductility/on-cooling-schematic.png}}
    \caption{Schematic diagrams of on-heating and on-cooling hot ductility tests performed at various temperatures in a simulated welding thermal cycle.}
\end{figure}

\begin{figure}
    \centering
    \includegraphics[width=4in]{figures/hot-ductility/schematic-hot-ducility-curves.png}
    \caption{Schematic hot ductility curves showing on-heating and on-cooling hot ductility behavior.}
    \label{fig:schematic-hot-ductility-curves}
\end{figure}

\subsection{Hot Ductility Evaluation Criteria}
In one of their early studies utilizing the Gleeble hot ductility test to evaluate a number of alloys, \citet{nippes_further_1957} classified the observed hot ductility responses of the various alloys into several categories, based on the shapes of the hot ductility curves: Classes H1 or H2 for the on-heating behavior and Classes C1, C2, or C3 for the on-cooling behavior. Schematic hot ductility curves illustrating the characteristics of each behavior class are shown in Figure~\ref{fig:nippes-criteria} and a brief text description for each is provided in Table~\ref{tab:nippes-classification}. Of the two on-heating behavior categories, Class H2 (Figure~\ref{fig:nippes-criteria}b) behavior was identified as being intrinsically sensitive to hot cracking. Class H1 behavior (Figure~\ref{fig:nippes-criteria}a) was not considered in and of itself indicative of a propensity for hot cracking and thus required the evaluation of on-cooling results to determine the material's susceptibility. With regard to the on-cooling categories, materials exhibiting Class C1 (Figure~\ref{fig:nippes-criteria}c) behavior were considered not sensitive to hot cracking. Class C2 and Class C3 behaviors (Figure~\ref{fig:nippes-criteria}d,e) were associated with a higher sensitivity to hot cracking, with Class C3 behavior indicating the greatest sensitivity. In particular, \citeauthor{nippes_further_1957} found that those materials in the study which were known to exhibit hot cracking, based on field experience, exhibited Class C3 behavior with on-cooling ductility of 40\% or less of the on-heating ductility. Thus, using the Nippes classification system, in most cases a material's on-cooling ductility response will be the major indicating factor regarding its hot cracking susceptibility. Materials exhibiting Class C3 on-cooling behavior with a low recovery of on-cooling ductility (0--40\% of on-heating ductility) can be identified as showing the greatest susceptibility to hot cracking.

% Nippes Criteria
%\bigskip
\begin{figure}[h]
\centering
\includegraphics[width=6in]{figures/nippes-criteria.png}
\caption{Classification of Hot Ductility Behavior for On-heating and On-Cooling Tests; in (c), (d), and (e), the solid line is the on-cooling curve and the dashed line is the on-heating curve.  From \citet[Fig.~66]{nippes_further_1957}.}
\label{fig:nippes-criteria}
\end{figure}

\begin{table}[h]
\caption{Classification of on-heating and on-cooling hot ductility responses based on the work of \citet{nippes_further_1957}. Schematic curves for each behavior class are depicted in Figure~\ref{fig:nippes-criteria}.}
\begin{tabular}{ lp{4in} }
\toprule
\textbf{Classification} & \textbf{Description} \\
\midrule
On-Heating Class H1 & On-heating ductility generally increases as temperature increases, followed by a sudden loss of ductility over a relatively narrow range as the temperature increases further toward the melting point. (Figure~\ref{fig:nippes-criteria}a) \\
\addlinespace
On-Heating Class H2 & On-heating ductility shows a gradual decrease over a wide temperature range as the temperature increases toward the melting point. (Figure~\ref{fig:nippes-criteria}b) \\
& \\
On-Cooling Class C1 & On-cooling ductility is the essentially same as on-heating ductility at all test temperatures. (Figure~\ref{fig:nippes-criteria}c) \\
\addlinespace
On-Cooling Class C2 & On-cooling ductility is the same as on-heating ductility at test temperatures of 2100\textdegree{}F or above, but is significantly lower at test temperatures in the range of 1800--2000\textdegree{}F. (Figure~\ref{fig:nippes-criteria}d) \\
\addlinespace
On-Cooling Class C3 & On-cooling ductility is lower than on-heating ductility at all test temperatures; severity of ductility decrease may change with on-cooling test temperature or with the peak temperature utilized for the on-cooling thermal cycle. (Figure~\ref{fig:nippes-criteria}e) \\
\bottomrule
\end{tabular}
\label{tab:nippes-classification}
\end{table}


Further investigation of hot ductility test criteria was undertaken in a review by \citet{yeniscavich_correlation_1970}. One of the criteria reviewed, attributed to Nippes, was the \gls{drr}. According to this criteria, crack-resistant materials show a rapid recovery of on-cooling ductility after exposure to the \gls{haz} peak temperature, while the on-cooling ductility for crack-sensitive materials recovers slowly and remains low even at test temperatures well below the peak temperature. Schematic curves illustrating the \gls{drr} for crack-resistant and crack-sensitive materials are shown in Figure~\ref{fig:drr-schematic}. Numerically, the \gls{drr} is determined by taking the ratio of the on-cooling ductility to the on-heating ductility at a specified temperature, typically the rapid-ductility-decrease temperature on the on-heating curve (the point on the curve immediately prior to the sudden drop in ductility).

\begin{figure}
\centering
\includegraphics[width=6in]{figures/yeniscavich-drr.png}
\caption{Schematic curves illustrating the on-cooling \acrshort{drr} criteria for crack-resistant and crack-sensitive materials.  From \citet[Fig.~2]{yeniscavich_correlation_1970}}
\label{fig:drr-schematic}
\end{figure}

Yeniscavich also proposed the \gls{zdr} as an improved indicator of propensity for hot cracking. The \gls{zdr} phenomenon corresponds to a finite temperature increment below the \gls{zdt} in which the on-cooling ductility remains zero, i.e. the ductility does not immediately increase once the on-cooling test temperature is below the \gls{zdt}. The physical significance of the \gls{zdr} is related to the fact that the HAZ must possess sufficient ductility to withstand the thermal strains imposed during welding, otherwise cracking will occur. Since the magnitude of these strains is small over the length scale of an HAZ, the HAZ ductility must be on the order of zero for hot cracking to be of concern. Thus, alloys which show a large \gls{zdr} (zero ductility over a wide on-cooling temperature range) are considered more vulnerable to hot cracking than alloys which show a narrow \gls{zdr} (see Figure~\ref{fig:zdr-schematic}).

\begin{figure}
\centering
\includegraphics[width=6in]{figures/zdr-schematic.png}
\caption[Schematic of Hot Ductility Curves Illustrating Different On-Cooling Behaviors According to the Zero Ductility Range Criteria.]{Schematic of Hot Ductility Curves Illustrating Different On-Cooling Behaviors According to the Zero Ductility Range Criteria: Crack-Sensitive (\gls{zdr}') and Crack-Resistant (\gls{zdr}).  From \citet[Fig.~4]{yeniscavich_correlation_1970}}
\label{fig:zdr-schematic}
\end{figure}


